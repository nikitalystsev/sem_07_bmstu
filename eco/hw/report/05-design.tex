\chapter{Практическая часть}

\includeimage
{empty-crossword.jpg}
{f}
{H}
{0.8\textwidth}
{Незаполненный кроссворд}

\textbf{По горизонтали:}

2. Вид олигополии когда на рынке представлены только два производителя конкретной продукции.
4. Модель базирующаяся на том, что: две фирмы производят однородный товар, им известна кривая рыночного спроса, они принимают решения о производстве независимо друг от друга и одновременно.
5. Для медленно растущих олигопольных рынков они очень высокие.
7. Ценовое … -- одна из форм сознательного параллелизма.
8. Олигополия, при котором фирмы согласуют своё поведение.
10. Модель описывающая поведение фирм на олигополистическом рынке, конкурирующих за счет изменения(в основном снижения) уровня цен на свою продукцию.
11. Вид олигополии при которой фирмы представляют схожие продукты, характеризующиеся обилием типов, сортов, размеров и т.д.
12. Важное значение для сохранения картеля играют … между участниками.

\textbf{По вертикали:}

1. В честь этого экономиста названа степень влияния фирмы на рыночные цены.
3. Рыночная структура, большая часть производства и продаж которой осуществляется небольшим числом сравнительно крупных предприятий.
4. Группа олигополистов, договаривающихся об определенных принципах установления цен, распределения долей рынка.
6. Олигополия, при котором фирмы действуют независимо.
9. Модель этой кривой была разработана в 1939 г. экономистами Полем Свизи (суизи), Хэллом и Хитчем.

\includeimage
{fill-crossword.jpg}
{f}
{H}
{0.8\textwidth}
{Заполненный кроссворд}
