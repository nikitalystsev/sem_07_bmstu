\chapter{Теоретическая часть}

\section{Олигополия}

Под термином <<Олигополия>> понимается тип рыночной структуры, при котором на рынке господствует несколько крупных фирм, каждая из которых способна оказывать влияние на рыночную цену своими собственными действиями \cite{info_oly3}. 

У олигополии выделяют следующие основные черты:

\begin{itemize}
	\item \textbf{небольшое число фирм, господствующих на рынке.} Немногочисленность продавцов, которые действуют на рынке, отличительная особенность олигополии, но не стоит считать, что компании на рынке олигополии можно пересчитать по пальцам. В олигополистической отрасли наряду с крупными часто действует немало мелких фирм. Однако на несколько ведущих компаний приходится столь большая часть суммарного оборота отрасли, что именно их деятельность определяет развитие событий. К олигополистическим структурам экономисты обычно относят те рынки, где несколько крупнейших продавцов (в разных странах за точку отсчета принято от 3 до 8 фирм) производят более пятидесяти процентов всей выпускаемой продукции;
	
	\item \textbf{однородность и дифференцированность олигополии.} Олигополия может быть однородной и дифференцированной. Многие промышленные товары, сталь, цинк, медь, алюминий, цемент, технический спирт и др. -- являются стандартизированной продукцией и производятся на олигополизированных территориях. Многие отрасли промышленности производят товары народного потребления: автомобили, сигареты и т. д. производятся дифференцированными олигополиями;
	
	\item \textbf{ограничение доступа на рынок.} К естественным преградам относятся эффект масштаба, который может сделать не-прибыльным сосуществование многих фирм на рынке, так как для этого нужны большие финансовые средства (речь идет об естественной олигополии) и патентование и лицензирование производственных технологий. Вместе с тем фирмами могут быть предприняты и стратегические действия, затрудняющие вступление новых фирм в отрасль. Например, фирмы могут угрожать конкурентам, что пере-полнят рынок товарами и снизят тем самым цены. Но время от времени приток новых фирм все-таки происходит \cite{info_oly3};
		
	\item \textbf{влияние каждой фирмы на рыночную цену.} Поведение фирм на олигополистическом рынке подвержено двум противоположным тенденциям. Во-первых, взаимозависимость фирм может привести их к желанию вступить в сговор друг с другом. Во-вторых, эти фирмы будут стремиться конкурировать таким образом, чтобы получать именно в свое распоряжение максимальную долю отраслевой прибыли. При этом важно понимать, что эти две линии поведения несовместимы. Чем больше фирмы конкурируют друг с другом ради получения большей доли отраслевой прибыли, тем меньше становится общая величина этой прибыли;
	
	\item \textbf{всеобщая взаимозависимость фирм.} Главная отличительная черта олигополии как специального типа строения рынка заключается во всеобщей взаимозависимости поведения предприятий-продавцов. Олигополист должен сделать (или принять) некоторые предположения о реакции своих соперников на принимаемые им решения, а также о реакции соперников на результаты своих решений. Зависимость
	поведения каждой фирмы от реакции конкурентов называется \textbf{олигополистической взаимосвязью} \cite{info_oly3}.
\end{itemize}

\section{Основные модели олигополии}

Общей модели поведения олигополиста при выборе оптимального объема производства, максимизирующего прибыль, не существует. 
Так как выбор зависит от поведения фирмы в ответ на изменения действий конкурентов, то могут возникать различные ситуации. 
В связи с этим различают следующие основные модели олигополии \cite{info_oly3}:

\begin{itemize}
	\item модель Курно;
	\item олигополия, основанная на тайном сговоре;
	
	\clearpage
	
	\item олигополия, не основанная на тайном сговоре (дилемма заключенного);
	\item молчаливый сговор: лидерство в ценах.
\end{itemize}

\subsection{Модель Курно}

Модель Курно (дуополии) была представлена в 1838 г. французским экономистом А.Курно. 

Дуополия -- это ситуация, когда на рынке конкурируют друг с другом только две фирмы. При
построении этой модели предполагается, что фирмы производят однородный товар и что известна рыночная кривая спроса. Максимизирующий прибыль объем производства фирмы 1 ($Q_1$) изменяется в зависимости от того, как, по ее мнению, будет расти объем производства фирмы 2 ($Q_2$). В результате каждая фирма строит свою кривую реакции (рис \ref{img:kyrno.png}) \cite{info_oly3}.

\includeimage
{kyrno.png}
{f}
{ht!}
{0.8\textwidth}
{Равновесие Курно}

Кривая реакции каждой фирмы говорит о том, сколько она будет производить при том или ином предполагаемом объеме производства своего конкурента. 
При равновесии каждая фирма устанавливает объем производства в соответствии со своей кривой реакции. 
Поэтому равновесный уровень объема производства находится на пересечении двух кривых реакции. 
Это равновесие называется равновесием Курно. 
При нем каждый дуополист устанавливает объем производства, который максимизирует его прибыль, при данном объеме производства своего конкурента. 
Равновесие Курно является примером того, что в теории игр называется равновесием Нэша (когда каждый игрок делает наилучшее, что можно, при заданных действиях оппонентов, в итоге — ни у одного игрока нет стимула, чтобы изменить свое поведение) \cite{info_oly3}.

\subsection{Олигополия, основанная на тайном сговоре}

Тайный сговор -- это фактическое соглашение между фирмами в отрасли с целью установления фиксированных цен и объемов производства. 
Такое соглашение называется картелем.
Во многих странах тайный сговор считается незаконным, а в Японии, например, он получил
большое распространение. 
К факторам, способствующим тайному сговору, относятся \cite{info_oly3}: 

\begin{itemize}
	\item наличие правовой базы;
	\item высокая концентрация продавцов;
	\item примерно одинаковые средние издержки у фирм в отрасли;
	\item невозможность проникновения новых фирм на рынок.
\end{itemize}


Предполагается, что при тайном сговоре каждая фирма будет выравнивать свои цены и при понижении, и при повышении цен. 
При этом фирмы производят однородную продукцию и имеют одинаковые средние издержки. 
Тогда при выборе оптимального объема производства, максимизирующего прибыль, олигополист ведет себя подобно чистому монополисту. 
Если две фирмы сговорились, то они строят кривую контракта (рис \ref{img:kr-kontr.png}).

\includeimage
{kr-kontr.png}
{f}
{H}
{0.8\textwidth}
{Кривая контракта при тайном сговоре}

Она показывает различные сочетания объемов производства двух фирм, которые максимизируют прибыль. 
Тайный сговор значительно выгоднее для фирм, чем не только совершенное равновесие, но и равновесие Курно, так как они будут производить меньше продукции, а цену установят выше.

\subsection{Олигополия, не основанная на тайном сговоре}

Если же тайного сговора нет (а в большинстве стран он является незаконным), то олигополисты при установлении цены сталкиваются с дилеммой заключенных. 
Это классический пример теории игр в экономике: двух заключенных обвинили в
совместном совершении преступления. 
Они находятся в отдельных тюремных камерах и не могут поддерживать связь друг с другом. 
Если оба сознаются — каждый получит срок наказания в 5 лет. 
Если оба не сознаются — дело не будет доведено до конца и каждый получит по 2 года. 
Если один сознается, а другой нет — то первый получит 1 год наказания, а другой — 10 лет. Матрица возможных результатов (рис \ref{img:dil-zakl.png}) \cite{info_oly3}:

\includeimage
{dil-zakl.png}
{f}
{H}
{0.8\textwidth}
{Дилемма заключенного}

Перед заключенными стоит дилемма: признаваться или не признаваться. 
Если бы они могли договориться, чтобы не признаваться, они бы не признались и получили бы по 2 года наказания. 
Но даже если бы такая возможность существовала, они не могут доверять друг другу. 
Если  один не признается, то он рискует, что второй этим воспользуется. 
Поэтому, чтобы ни делал первый, второму всегда выгоднее признаться. 
Тогда, вероятнее всего, признаются оба и пойдут в тюрьму на 5 лет.

Олигопольные фирмы часто сталкиваются с дилеммой заключенных. 
Предположим, что две фирмы — единственные продавцы на рынке. 
Они сталкиваются с дилеммой: какую цену установить: высокую или низкую? 
Если договорятся и оба установят высокую цену, то получат по 20 млн руб., если установят оба относительно низкую цену — то по 15 млн руб., если одна из фирм установит высокую цену, а другая — низкую, то первая фирма получит 10 млн руб., а вторая — 30 (за счет первой), т. е. есть стремление к обману. 
Матрица возможных результатов (рис \ref{img:dil-zakl2.png}) \cite{info_oly3}:

\includeimage
{dil-zakl2.png}
{f}
{H}
{0.8\textwidth}
{Дилемма заключенного для олигополий}

Очевидно, что каждой фирме выгодно назначить относительно низкую цену, независимо от того, как поступит конкурент. 
Поэтому фирмы назначат низкие цены и получат по 15 млн руб. 
Дилемма заключенных объясняет жесткость цены при олигополии.


\subsection{Молчаливый сговор: лидерство в ценах}

Существует еще одна модель поведения олигополистов, основанная на молчаливом тайном соглашении: это <<лидерство в ценах>>, когда доминирующая на рынке фирма меняет цену,
а все другие следуют этому изменению. 
Ценовому лидеру с молчаливого согласия остальных отводится ведущая роль в установлении отраслевых цен. 
Ценовой лидер может объявить об изменении цены, и, если его расчет верен, то остальные фирмы также увеличивают цены. 
В результате отраслевая цена изменяется без тайного сговора \cite{info_oly3}. 