\chapter*{ВВЕДЕНИЕ}
\addcontentsline{toc}{chapter}{ВВЕДЕНИЕ}

Олигополия есть самый распространенный тип рыночной структуры в российской экономике \cite{info_oly}. Примерами могут выступать различные отрасли: нефтегазовый комплекс, где главными комнурентами выступают <<Газпром>>, <<Роснефть>> и <<Лукойл>>, рынок мобильной связи с ключевыми игроками в виде <<МТС>>, <<Мегафон>>, <<Билайн>> и <<Теле 2>>, рынок маркетплейсов, где доминируют <<Озон>> и <<Wildberries>>.

На международном рынке также можно выделить множество примеров олигополии: рынок игровых консолей, где <<PlayStation>>, <<Xbox>>, <<Nintendo>> занимают лидирующие позиции, авиационная промыщленность, представленная компаниями <<Boeing>> и <<Airbus>>, чьи самолеты занимают большую часть в авиапарках мира, рынок производителей процессоров для персональных компьютеров, где <<Intel>> и <<AMD>> занимают лидирующие позиции \cite{info_oly2}.

Учитывая популярность данного типа рыночной структуры, необходимо понимать, что он из себя представляет.

Целью данной работы является раскрытие понятия <<Олигополия>> и описание ее основных моделей. 

Для достижения поставленной цели необходимо решить следующие задачи: 

\begin{enumerate}
	\item описать термин <<Олигополия>>;
	\item описать модели олигополии;
	\item составить кроссворд.
\end{enumerate}