\chapter{Распределение Пуассона}

Говорят, что случайная величина $X$ имеет \textit{распределение Пуассона} с параметром $\lambda > 0$, если $X$ принимает значения $0, 1, 2, \dots$ с вероятностями:

\begin{equation}
	P\{X=k\} = \frac{\lambda^k}{k!} e^{-\lambda}, k \in \mathbb{R}_0.
\end{equation}

При этом функция распределения~$F(x)$ равна:

\begin{equation}
	F(x) =
	\begin{cases}
		0,  & \quad x < a;\\
		\displaystyle\frac{x - a}{b - a}, & x \in [a, b];\\
		1,  & \quad x > b.
	\end{cases}
\end{equation}

Обозначение: $X \sim R[a, b]$.