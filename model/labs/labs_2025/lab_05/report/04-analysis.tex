\chapter{Техническое задание}

В информационный центр приходят клиенты через интервалы времени $10\pm2$~минуты.
Если все три имеющихся оператора заняты, клиенту отказывают в обслуживании.
Операторы имеют разную производительность и могут обеспечивать обслуживание среднего запроса за $20\pm5$,~$40\pm10$,~$40\pm20$~минут. 
Клиенты стремятся занять свободного оператора с максимальной производительностью.
Полученные запросы сдаются в приемные накопители. 
Откуда они выбираются на обработку. 
На первый компьютер --- запросы от первого и второго операторов, на
второй компьютер --- от третьего. 
Время обработки на первом и втором компьютере равны соответственно $15$~и~$30$~минутам. 
Промоделировать процесс обработки $300$~запросов с целью определить вероятность отказа.

Необходимо для этого создать концептуальную модель и модель в терминах СМО, определить Эндогенные и Экзогенные переменные и уравнения модели. 
Выбрать любой алгоритм протяжки модельного времени. В случае выбора принципа $\Delta t$ за единицу системного времени выбрать $0.01$ минуты.