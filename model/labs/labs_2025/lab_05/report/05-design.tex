\chapter{Концептуальная модель и модель в терминах СМО информационного центра}

На рисунке \ref{img:queue-system-diagram} представлена схема концептуальной модели.

На рисунке \ref{img:queue-system-diagram} представлена схема модели в терминах СМО.

\includeimage
{conceptual-diagram} % Имя файла без расширения (файл должен быть расположен в директории inc/img/)
{f} % Обтекание (без обтекания)
{h} % Положение рисунка (см. figure из пакета float)
{1\textwidth} % Ширина рисунка
{Схема концептуальной модели} % Подпись рисунка

\includesvgimage
{queue-system-diagram} % Имя файла без расширения (файл должен быть расположен в директории inc/img/)
{f} % Обтекание (без обтекания)
{h} % Положение рисунка (см. figure из пакета float)
{1\textwidth} % Ширина рисунка
{Схема модели в терминах СМО} % Подпись рисунка

\chapter{Эндогенные и Экзогенные переменные и уравнения имитационной модели}

Эндогенные переменные:

\begin{itemize}
	\item время обработки задания $i$-ым оператором;
	\item время решения задания на $j$-ом компьютере.
\end{itemize}

Экзогенные переменные:
\begin{itemize}
	\item $C_{\text{обсл}}$ --- число обслуженных клиентов;
	\item $C_{\text{отк}}$ --- число клиентов, получивших отказ.
\end{itemize}

Вероятность отказа рассчитывается по формуле~\ref{eq:01}, которая описывает уравнение модели:

\begin{equation}\label{eq:01}
	P_{\text{отказа}} = \frac{C_{\text{отк}}}{C_{\text{отк}} + C_{\text{обсл}}}.
\end{equation}