\chapter{Законы распределения}

Время прихода заявки из источника информации и время освобождения обслуживающего аппарата может изменяться по одному из следующих пяти законов распределения со следующими формулами вычисления моментов времени $t_i$:

\begin{itemize}
	\item равномерное распределение;
	
	\begin{equation}
		t_i = a + (b - a)R_i
	\end{equation}
	
	\item распределение Эрланга;
	
	\begin{equation}
		t_i = -\frac{1}{k\lambda} \sum_{i=1}^{k} \ln(1 - R_i)
	\end{equation}
	
	\item распределение Пуассона --- относится к дискретному с мат. ожиданием и дисперсией, равными $\lambda > 0$. 
	Для генерирования пуассоновских переменных используется метод точек, в основе которого лежит генерирование случайной переменной $R_i$, равномерно распределенной на [0, 1], до тех пор, пока не будет выполнено следующее соотношение:
	
	\begin{equation}
		\prod_{i=0}^{x} R_i \ge e^{-\lambda} > \prod_{i=0}^{x+1} R_i
	\end{equation}
	
	\item экспоненециальное распределение;
	
	\begin{equation}
		t_i = \frac{1}{\lambda} \ln(1 - R_i)
	\end{equation}
	
	\item нормальное распределение. Для упрощения $n = 12$
	
	\begin{equation}
		t_i = \sigma \sqrt\frac{12}{n}(\sum\limits_{i=1}^{n} R_i - \frac{n}{2}) + m,
	\end{equation}
	
\end{itemize}


\chapter{Принцип $\Delta t$}
Принцип $\Delta t$ заключается в последовательном анализе состояний всех блоков в момент времени $t + \Delta t$ по заданному состоянию блоков в момент времени $t$. 
При этом новое состояние блоков определяется в соответствии с их алгоритмическим описанием с учетом действующих случайных факторов. 
В результате этого анализа принимается решение о том, какие общесистемные события должны имитироваться программой на данный момент времени.

Основной недостаток принципа $\Delta t$ заключается в значительных затратах вычислительных ресурсов, а при недостаточно малом $\Delta t$ появляется опасность пропуска отдельных событий в системе, исключающая возможность получения правильных результатов при моделировании.

\chapter{Событийный принцип}

Состояния отдельных устройств изменяется в дискретные моменты времени, совпадающие с моментами поступления сообщений в систему, окончания реализации задания, поэтому моделирование и продвижение текущего времени в системе удобно проводить, используя событийных принцип.

При использовании данного принципа состояние всех блоков имитационной модели анализируется лишь в момент появления какого-либо события. 
Момент наступления следующего события определяется минимальными значениями из списка будущих событий, представляющего собой совокупность моментов ближайшего изменения состояний каждого из блока системы.