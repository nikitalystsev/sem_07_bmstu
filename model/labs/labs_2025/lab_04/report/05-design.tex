% посмотреть как можно редактировать интервалы вывода графиков в зависимости от закона распределения

\chapter{Равномерное распределение}

Говорят, что непрерывная случайная величина $X$ имеет \textit{равномерное распределение} на отрезке~$[a,~b]$ ($a, b \in (-\infty, +\infty)$) , если ее функция плотности распределения~$f(x)$ равна:

\begin{equation}
	f(x) =
	\begin{cases}
		\displaystyle\frac{1}{b - a}, & x \in [a, b];\\
		0,  & \quad \text{иначе}.
	\end{cases}
\end{equation}

При этом функция распределения~$F(x)$ равна:

\begin{equation}
	F(x) =
	\begin{cases}
		0,  & \quad x < a;\\
		\displaystyle\frac{x - a}{b - a}, & x \in [a, b];\\
		1,  & \quad x > b.
	\end{cases}
\end{equation}

Обозначение: $X \sim R[a, b]$.

Математическое ожидание:

\begin{equation}
	M[X] = \frac{a + b}{2}
\end{equation}

Дисперсия: 

\begin{equation}
	D[X] = \frac{(b - 2)^2}{12}
\end{equation}

\chapter{Распределение Эрланга} % проверил

Говорят, что непрерывная случайная величина $X$ имеет \textit{распределения Эрланга} c параметрами $\lambda$ и $k$ ($\lambda > 0; k = 0, 1, 2, ...$) если ее плотность распределения~$f_k(x)$ равна:

\begin{equation}
	f_k(x) =
	\begin{cases}
		 \frac{\lambda^{k} \cdot x^{k-1}}{(k-1)!} \cdot e^{-\lambda \cdot x}, & x \ge 0;\\
		 0,  & \text{иначе}.
	\end{cases}
\end{equation}


При этом функция распределения~$F_k(x)$ равна:

\begin{equation}
	F_k(x) =
	\begin{cases}
		 1 - e^{-\lambda \cdot x} \cdot \sum_{i = 0}^{k-1} \frac{(\lambda \cdot x)^i}{i!}, & x \ge 0;\\
		 0,  & \text{иначе}.
	\end{cases}
\end{equation}

\textbf{Физический смысл распределения} --- вероятность того, что в пуассоновском процессе $k$-e событие произойдёт через заданный промежуток времени.

\textbf{Физический смысл параметров:}

\begin{itemize}
	\item  $\lambda$ --- среднее количество событий за фиксированный промежуток времени (интенсивность);
	
	\item $k$ --- количество событий.
\end{itemize}

% распределение Эрланга — это распределение суммы k независимых экспоненциально распределённых величин.

% иначе говоря, это распределение времени до k-го события пуассоновского процесса с интенсивностью λ. 

% распределения Эрланга и Пуассона дополняют друг друга: в то время как распределение Пуассона подсчитывает события, происходящие за фиксированный промежуток времени, распределение Эрланга подсчитывает время до наступления фиксированного количества событий. При k = 1 распределение упрощается до экспоненциального распределения.

Математическое ожидание:

\begin{equation}
	M[X] = \frac{k}{\lambda}
\end{equation}

Дисперсия: 

\begin{equation}
	D[X] = \frac{k}{\lambda^2}
\end{equation}

\chapter{Распределение Пуассона} % проверил

Говорят, что  дискретная случайная величина $X$ имеет \textit{распределение Пуассона} с параметром $\lambda > 0$, если $X$ принимает значения $0, 1, 2, \dots$ с вероятностями:

\begin{equation}
	P\{X=k\} = \frac{\lambda^k}{k!} e^{-\lambda}, k \in \mathbb{N}_0.
\end{equation}

Функция распределения: 

\begin{equation}
	F(x) = \sum\limits_{i = 0}^{x} \frac{\lambda^i}{i!} e^{-\lambda}
\end{equation}

\textbf{Физический смысл распределения} --- вероятность наступления заданного числа событий, произошедших за фиксированное время, при условии, что данные события происходят с некоторой фиксированной средней интенсивностью и независимо друг от друга.

\textbf{Физический смысл параметра $\lambda$} --- среднее количество событий за фиксированный промежуток времени (интенсивность).

% 3 критерия: события независимы, среднее количество событий в единицу времени постоянно, события не происходят одновременно

% это закон редких событий: наблюдаем события, которые редки сами по себе (например, звонок отдельного человека в колл-центр), но когда мы наблюдаем большое число таких событий (например, сидя в колл-центре). Количество звонков получается своего рода сумма большого количества редких событий

% lambda --- произведение вероятности того, что в единицу времени (мб минуту, секунду) произойдет событие на общую длину временного интервала [так как события независимы].

% пример использования: например известно, сколько раз в месяц в среднем падает сайт. Это lambda. Чтобы посчитать, с какой вероятностью сайт не упадет ни разу за месяц (k = 0), нужно использовать формулу распределения Пуассона.

Обозначение: $X \sim \Pi(\lambda)$.

Математическое ожидание:

\begin{equation}
	M[X] = \lambda
\end{equation}

Дисперсия: 

\begin{equation}
	D[X] = \lambda
\end{equation}

\chapter{Экспоненциальное распределение} % проверил

Говорят, что  непрерывная случайная величина $X$ имеет \textit{экспоненциальное (показательное) распределение} с параметром $\lambda > 0$, если ее функция плотности распределения~$f(x)$ имеет вид:

\begin{equation}
	f(x) =
	\begin{cases}
		\lambda \exp(-\lambda x), & x \ge 0;\\
		0,  & \text{иначе}.
	\end{cases}
\end{equation}

Функция распределения: 

\begin{equation}
	F(x) = 
	\begin{cases}
		1 - \exp(-\lambda x), & x > 0;\\
		0,  & \text{иначе}.
	\end{cases}
\end{equation}

\textbf{Физический смысл распределения} --- вероятность того, что в пуассоновском процессе следующее событие произойдёт через заданный промежуток времени.

\textbf{Физический смысл параметра $\lambda$} --- среднее количество событий за фиксированный промежуток времени (интенсивность).

% критерии: события независимы

% экспоненциальное распределение моделирует время или расстояние между событиями в пуассоновском процессе.

% это распределение моделирует время между двумя последовательными свершениями одного и того же события.

% lambda --- произведение вероятности того, что в единицу времени (мб минуту, секунду) произойдет событие на общую длину временного интервала [так как события независимы].

Обозначение: $X \sim \mathrm{Exp}(\lambda)$.

Математическое ожидание:

\begin{equation}
	M[X] = \frac{1}{\lambda}
\end{equation}

Дисперсия: 

\begin{equation}
	D[X] = \frac{1}{\lambda^2}
\end{equation}

\chapter{Нормальное распределение}

Говорят, что  непрерывная случайная величина $X$ имеет \textit{нормальное распределение} с параметром $\mu$ и $\sigma^2$, если ее функция плотности распределения~$f(x)$ имеет вид:

\begin{equation}
	f(x) = \frac{1}{\sqrt{2\pi} \sigma} \exp\Bigg( -\frac{(x - \mu)^2}{2\sigma^2} \Bigg), \space x \in \mathbb{R}_0
\end{equation}

Функция распределения: 

\begin{equation}
	F(x) = \frac{1}{\sigma \sqrt{2\pi}} \int_{-\infty}^{x} e^{ -\frac{(t - \mu)^2}{2\sigma^2} } \, dt
\end{equation}

\textbf{Геометрический смысл параметров:}

\begin{itemize}
	\item  $\mu$ --- отвечает за локализацию точки максимума функции $f(x)$ (и оси симметрии);
	
	\item $\sigma$ --- отвечает за концентрацию значений в районе точки $X=\mu$. Чем меньше $\sigma$, тем выше концентрация.
\end{itemize}

Обозначение: $X \sim \mathrm{N}(\mu, \sigma^2)$.

Математическое ожидание:

\begin{equation}
	M[X] = \mu
\end{equation}

Дисперсия: 

\begin{equation}
	D[X] = \sigma^2
\end{equation}