\chapter{Равномерное распределение}

Говорят, что непрерывная случайная величина $X$ имеет \textit{равномерное распределение} на отрезке~$[a,~b]$, если ее функция плотности распределения~$f(x)$ равна:

\begin{equation}
	f(x) =
	\begin{cases}
		\displaystyle\frac{1}{b - a}, & x \in [a, b];\\
		0,  & \quad \text{иначе}.
	\end{cases}
\end{equation}

При этом функция распределения~$F(x)$ равна:

\begin{equation}
	F(x) =
	\begin{cases}
		0,  & \quad x < a;\\
		\displaystyle\frac{x - a}{b - a}, & x \in [a, b];\\
		1,  & \quad x > b.
	\end{cases}
\end{equation}

Обозначение: $X \sim R[a, b]$.

\chapter{Распределение Пуассона}

Говорят, что  дискретная случайная величина $X$ имеет \textit{распределение Пуассона} с параметром $\lambda > 0$, если $X$ принимает значения $0, 1, 2, \dots$ с вероятностями:

\begin{equation}
	P\{X=k\} = \frac{\lambda^k}{k!} e^{-\lambda}, k \in \mathbb{N}_0.
\end{equation}

Функция распределения: 

\begin{equation}
	F(x) = \sum\limits_{i = 0}^{x} \frac{\lambda^i}{i!} e^{-\lambda}
\end{equation}


Обозначение: $X \sim \Pi(\lambda)$.


\chapter{Экспоненциальное распределение}

Говорят, что  непрерывная случайная величина $X$ имеет \textit{экспоненциальное (показательное) распределение} с параметром $\lambda > 0$, если ее функция плотности распределения~$f(x)$ имеет вид:

\begin{equation}
	f(x) =
	\begin{cases}
		\lambda \exp(-\lambda x), & x \ge 0;\\
		0,  & \text{иначе}.
	\end{cases}
\end{equation}

Функция распределения: 

\begin{equation}
	F(x) = 
	\begin{cases}
		1 - \exp(-\lambda x), & x > 0;\\
		0,  & \text{иначе}.
	\end{cases}
\end{equation}

Обозначение: $X \sim \mathrm{Exp}(\lambda)$.

\chapter{Нормальное распределение}

Говорят, что  непрерывная случайная величина $X$ имеет \textit{нормальное распределение} с параметром $\mu$ и $\sigma^2$, если ее функция плотности распределения~$f(x)$ имеет вид:

\begin{equation}
	f(x) = \frac{1}{\sqrt{2\pi} \sigma} \exp\Bigg( -\frac{(x - \mu)^2}{2\sigma^2} \Bigg), \space x \in \mathbb{R}_0
\end{equation}

Функция распределения: 

\begin{equation}
	F(x) = \frac{1}{\sigma \sqrt{2\pi}} \int_{-\infty}^{x} e^{ -\frac{(t - \mu)^2}{2\sigma^2} } \, dt
\end{equation}

Обозначение: $X \sim \mathrm{N}(\mu, \sigma^2)$.