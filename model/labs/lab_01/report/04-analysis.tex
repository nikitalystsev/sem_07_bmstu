\chapter{Равномерное распределение}

Случайная величина $X$ имеет \textit{равномерное распределение} на отрезке~$[a,~b]$, если ее плотность распределения~$f(x)$ равна:

\begin{equation}
	f(x) =
	\begin{cases}
		\displaystyle\frac{1}{b - a}, & \quad \text{если } a \leq x \leq b;\\
		0,  & \quad \text{иначе}.
	\end{cases}
\end{equation}

При этом функция распределения~$F(x)$ равна:

\begin{equation}
	F(x) =
	\begin{cases}
		0,  & \quad x < a;\\
		\displaystyle\frac{x - a}{b - a}, & \quad a \leq x \leq b;\\
		1,  & \quad x > b.
	\end{cases}
\end{equation}

Обозначение: $X \sim R[a, b]$.