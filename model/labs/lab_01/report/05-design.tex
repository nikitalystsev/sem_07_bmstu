\chapter{Распределение Эрланга}

Случайная величина $X$ имеет \textit{распределения Эрланга} c параметрами $\lambda$ и $k$ ($(\lambda \geqslant 0; k = 0, 1, 2, ...)$, $x \geqslant 0$) если ее плотность распределения~$f_k(x)$ равна:

\begin{equation}
	\begin{aligned}
		f_k(x) = \frac{\lambda \cdot (\lambda \cdot x)^{k}}{k!} \cdot e^{-\lambda \cdot x}.
	\end{aligned}
\end{equation}

При этом функция распределения~$F_k(x)$ равна:

\begin{equation}
	\begin{aligned}
		F_k(x) = 1 - e^{-\lambda \cdot x} \cdot \sum_{i = 0}^{k} \frac{(\lambda \cdot x)^i}{i!}.
	\end{aligned}
\end{equation}



