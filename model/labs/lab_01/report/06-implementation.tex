\chapter{Технологический раздел}

В данном разделе будут перечислены требования к программному обеспечению, средства реализации и листинги кода.

\section{Требования к программному обеспечению}

К программе предъявляется ряд требований:

\begin{itemize} [label=--]
	\item мощность алфавита не должна превышать 64 символа;
	\item программа должна предоставлять возможность шифрования и расшифровки произвольного файла, а также текстового сообщения;
	\item программа должна корректно работать с пустым однобайтовым файлом;
	\item программа должна уметь обрабатывать файл архива.
\end{itemize}

\section{Средства реализации}

В качестве языка программирования для этой лабораторной работы был выбран $C++$ \cite{pl}.

\section{Сведения о модулях программы}

Программа состоит из трех модулей: 

\begin{itemize}
	\item \texttt{main.cpp} --- файл, содержащий точку входа в программу;
	\item \texttt{enigma.cpp} --- модуль, реализующий шифровальную машину <<Энигма>>;
	\item \texttt{encoder.cpp} --- модуль, реализующий -кодировщик.
\end{itemize}

\section{Тестирование}

\begin{table}[H]
	\begin{center}
		\begin{threeparttable}
			\caption{Функциональные тесты}
			\label{tbl:cmpResponseTimeByRequests_s_withoutCache}
			\begin{tabular}{|r|r|}
				\hline
				\bfseries \makecell{Входная строка} & \bfseries \makecell{Шифрованная строка} \\
				\hline
				HELLO & XFPRG  \\ 
				\hline
				XFPRG & HELLO  \\ 
				\hline
				HEL LO & XFP RG  \\ 
				\hline
				XFP RG & HEL LO  \\ 
				\hline
			\end{tabular}
		\end{threeparttable}
	\end{center}
\end{table}

Все тесты были успешно пройдены. Тесты с файлами были также успешно пройдены.

\clearpage

\section{Реализации алгоритмов}

В листинге \ref{lst:enigma-declaration.txt} представлено объявление класса, реализующего шифровальную машину <<Энигма>>.

\includelisting
{enigma-declaration.txt} % Имя файла с расширением (файл должен быть расположен в директории inc/lst/)
{Объявление класса, реализующего шифровальную машину <<Энигма>>} % Подпись листинга

\clearpage

В листинге \ref{lst:encrypt-char-by-code-impl.txt} представлена реализация алгоритма шифрования символа.

\includelisting
{encrypt-char-by-code-impl.txt} % Имя файла с расширением (файл должен быть расположен в директории inc/lst/)
{Реализация шифрования символа} % Подпись листинга

В листинге \ref{lst:encrypt-impl.txt} представлена реализация алгоритма шифрования символа c использованием кодировщика.

\includelisting
{encrypt-impl.txt} % Имя файла с расширением (файл должен быть расположен в директории inc/lst/)
{Реализация шифрования символа с использованием кодировщика} % Подпись листинга

В листингах \ref{lst:encrypt-string-impl-part1.txt}  и \ref{lst:encrypt-string-impl-part2.txt} представлена реализация шифрования строки.

\includelisting
{encrypt-string-impl-part1.txt} % Имя файла с расширением (файл должен быть расположен в директории inc/lst/)
{Реализация шифрования строки (начало)} % Подпись листинга

\includelisting
{encrypt-string-impl-part2.txt} % Имя файла с расширением (файл должен быть расположен в директории inc/lst/)
{Реализация шифрования строки (конец)} % Подпись листинга

В листингах \ref{lst:encrypt-file-impl-part1.txt}  и \ref{lst:encrypt-file-impl-part2.txt} представлена реализация шифрования произвольного файла.

\includelisting
{encrypt-file-impl-part1.txt} % Имя файла с расширением (файл должен быть расположен в директории inc/lst/)
{Реализация шифрования произовльного файла (начало)} % Подпись листинга

\includelisting
{encrypt-file-impl-part2.txt} % Имя файла с расширением (файл должен быть расположен в директории inc/lst/)
{Реализация шифрования произовльного файла (конец)} % Подпись листинга

%\includelisting
%{encrypt-string-impl-part2.txt} % Имя файла с расширением (файл должен быть расположен в директории inc/lst/)
%{Реализация шифрования строки (конец)} % Подпись листинга

%В листингах \ref{lst:search-tree-part1.txt} и \ref{lst:search-tree-part2.txt} представлена реализация функции поиска в ДДП и АВЛ-дереве.
%
%\includelisting
%{search-tree-part1.txt} % Имя файла с расширением (файл должен быть расположен в директории inc/lst/)
%{Реализация функции поиска в ДДП и АВЛ-дереве (начало)} % Подпись листинга
%
%\includelisting
%{search-tree-part2.txt} % Имя файла с расширением (файл должен быть расположен в директории inc/lst/)
%{Реализация функции поиска в ДДП и АВЛ-дереве (конец)} % Подпись листинга

\section*{Вывод}

В данном разделе были перечислены требования к программному обеспечению, средства реализации и листинги кода.

    