\chapter{Учет персональных черт}

В литературе существует масса различных способов моделирования личности робота. Рассмотрим некоторые.

\section{Таксономии личностей на основе набора черт}

В \cite{Recchiuto2023SoftwareFramework} предлагается таксономия трех черт из <<Большой пятерки>>, а именно экстраверсия, добросовестность и доброжелательность. 
Данная модель синтетических личностей на основе трёх независимых психологических измерений, где личность представлена как вектор в трёхмерном пространстве: % (формула \ref{equ:personality}):

\begin{equation}
	\label{equ:personality}
	Personality = W_eE + W_aA + W_cC, 
\end{equation}

где $E$, $A$ и $C$ соответствуют векторам трёх осей: экстраверсии, доброжелательности и сознательности. 
$W_e$, $W_a$ и $W_c$ — это длины трёх одномерных векторов, которые показывают, насколько выражена конкретная черта (в диапазоне [-1,1], где 0 означает нейтральность \cite{Nardelli2025EmpatheticCompanions})

Это позволяет описывать не фиксированные типы, а <<бесконечное множество вариаций личностей>>~\cite{}, моделируя их как линейную комбинацию трёх базовых черт.

\section{PERSONAGE}

\textbf{PERSONAGE (PERSONAlity GEnerator)} --- это первая~\cite{} система генерации языка, способная параметрически моделировать черты личности <<Большой пятёрки>> в речи, прежде всего экстраверсию.
Она строится на основе психолингвистических корреляций между языковыми особенностями и личностными чертами, описанными в психологии.

Она включает следующие модули.

\begin{itemize}
	\item Content Planner --- выбирает, о чём говорить (например, положительные или отрицательные аспекты).
	
	\item Sentence Planner --- решает, как это сказать (структура фраз, синтаксис, вставки вроде you know).
	
	\item Surface Realizer --- создаёт готовую текстовую реплику.
\end{itemize}

Авторы разработали архитектуру, где каждая стадия генерации (content planning, sentence planning, realization) управляется набором из 29 параметров, связанных с экстраверсией.

PERSONAGE умеет менять стиль речи под конкретную личность, сохраняя смысл и цель высказывания. Она создаёт фразы, где можно явно почувствовать <<характер>> говорящего --- от сдержанного и аналитичного до общительного и эмоционального.

Система обучается на психологических данных и может генерировать рекомендации с разным уровнем <<экстравертности>> речи.