\chapter{Подходы к описанию персональных черт}

Способ описания личности человека путём выделения некоторого набора черт занимает особое место в психологии личности, поскольку позволяет систематически характеризовать индивидуальные различия между людьми. 

\textbf{Черта личности (персональная черта)} --- это устойчивая предрасположенность человека реагировать, думать и чувствовать определённым образом, то есть проявлять характерные типы поведения, эмоций и мыслей в разных ситуациях \cite{PervinJohn2001Personality}.

\section{Личностные черты по Г.Оллпорту}

Г. Оллпорт разделил персональные черты человека на три уровня \cite{PervinJohn2001Personality}:

\begin{itemize}
	\item \textbf{кардинальные} --- представляют собой доминирующие диспозиции, которые пронизывают всю жизнь человека и проявляются практически во всех его действиях. Они определяют общий стиль поведения личности и становятся её ведущей характеристикой;
	
	\item \textbf{центральные} --- основные характеристики, описывающие поведение человека в повседневной жизни. К ним относятся такие качества, как честность, доброта, общительность, настойчивость. Эти черты формируют основу личности и определяют привычные способы взаимодействия с окружающими;
	
	\item \textbf{вторичные} --- являются менее устойчивыми и проявляются в ограниченных ситуациях. Они отражают индивидуальные предпочтения, привычки и вкусы, которые могут меняться в зависимости от обстоятельств.
\end{itemize}

Предложенная Оллпортом классификация позволяет рассматривать личность как систему иерархически организованных черт, различающихся по степени выраженности и влияния на поведение.

\clearpage

\section{Трехфакторная теория Ганса Ю. Айзенка}

В результате дважды проведенного факторного анализа Aйзенк выделил 3 базовых параметра личности, каждый  имеющий под собой биологическую основу: 

\begin{itemize}
	\item экстраверсия-интроверсия;
	
	\item невротизм;
	
	\item психотизм.
\end{itemize}

 Айзенк также разработал множество вопросников для измерения личности по этим параметрам: Личностный список модели, Личностный список Айзенка, Личностный вопросник Айзенка \cite{PervinJohn2001Personality}.

\section{Подход Раймонда Б. Кэттела}

В рамках своей теории он предложил две классификации личностных черт. Согласно первой, черты делятся на \cite{PervinJohn2001Personality}:
 
\begin{itemize}
	\item \textbf{черты способностей} --- характеризуют индивидуальные навыки и умения, обеспечивающие эффективность деятельности. Примером такой черты является интеллект;
	
	\item \textbf{черты темперамента} --- отражают эмоциональные особенности и стиль поведения человека — скорость реакции, уровень эмоциональной возбудимости, склонность к импульсивности или рассудительности;
	
	\item \textbf{динамические черты} --- описывают мотивационную сферу личности, то есть направленность и силу стремлений, а также значимые для человека цели.
\end{itemize}

Согласно второй, черты подразделяются на \cite{PervinJohn2001Personality}: 

\begin{itemize}
	\item \textbf{поверхностные черты} --- наблюдаемые особенности поведения, которые внешне кажутся взаимосвязанными, но не обязательно имеют общую причину. Они выявляются субъективными методами, например опросами и самооценками;
	
	\item \textbf{черты темперамента} --- представляют собой фундаментальные структуры личности, стоящие за наблюдаемыми проявлениями. Они выявляются с помощью факторного анализа и отражают реальные взаимосвязи между различными аспектами поведения. Именно глубинные черты, по Кэттеллу, образуют основу личности и служат её «строительными блоками».
\end{itemize}

Основное внимание исследователь уделял разработке опросников, в частности знаменитого теста 16PF, однако стремился продемонстрировать, что те же факторы выявляются и при использовании оценок внешних наблюдателей, и объективных методик.

\section{Большая пятерка}

\textbf{Большая пятерка} --- эмпирически выведенный (в результате процедуры факторного анализа) набор относительно независимых черт человека\cite{big_five_traits}. 

Термин <<Большая пятерка>> был введен  Л. Голдбергом в своей работе 1981 года \cite{goldberg1981language} как название для выделенных 5 черт личности:

\begin{itemize}
	\item \textbf{экстраверсия} --- показывает, насколько человек любит общение, полон энергии и положительных эмоций. Экстраверты активны, общительны, любят внимание и выражение своей позиции; слишком высокий уровень может проявляться навязчивостью. Низкие показатели отражают интроверсию -- спокойствие, самостоятельность, низкую потребность во внешней стимуляции; крайние значения могут указывать на трудности в социальных контактах \cite{big_five_traits};
	
	\item \textbf{доброжелательность} --- отражает просоциальную направленность личности. Люди с высокими показателями мягкие, терпимые, ориентированы на интересы группы, но при чрезмерной выраженности рискуют потерять индивидуальность. Низкие показатели характеризуют эгоцентричных, холодных и критичных людей, сосредоточенных на личной выгоде, однако способных к объективным решениям. \cite{radyuk2016bigfive};
	
	\item \textbf{добросовестность} --- показывает, насколько добросовестно человек выполняет свои обязанности, его целеустремленность, организованность, мотивированность. Высокие оценки по шкале являют надежную, пунктуальную, самодисциплинированную, педантичную личность. Низкие оценки - ленивую, беспечную, слабовольную, нецелеустремленную \cite{big_five_traits};
	
	\item \textbf{нейротизм} --- склонность испытывать негативные эмоции. Шкала отражает уровень эмоциональной стабильности личности. Высокие показатели указывают на повышенную реактивность и тревожность, тогда как низкие — на спокойное отношение к жизненным ситуациям \cite{big_five_traits};
	
	\item \textbf{открытость опыту} --- отражает интерес человека к новому — идеям, людям, местам. Высокие показатели связаны с творчеством, развитым воображением и тягой к новизне, низкие --- с консерватизмом, исполнительностью и предпочтением привычного. \cite{big_five_traits}.
\end{itemize}

Каждая из вышеперечисленных черт включает 6 дополнительных, более конкретных аспектов, без которых понимание основных факторов было бы неполным \cite{big_five_traits} (см. рисунок~\ref{img:big-five-traits}).

\includeimage
{big-five-traits} % Имя файла без расширения (файл должен быть расположен в директории inc/img/)
{f} % Обтекание (без обтекания)
{h} % Положение рисунка (см. figure из пакета float)
{1\textwidth} % Ширина рисунка
{Структура большой пятерки} % Подпись рисунка

Существует множество способов измерения Большой пятерки. 
Наиболее популярным является коммерческий опросник NEO PI-R, состоящий из 240 вопросов. 
Существуют также свободно доступные аналоги этого теста, такие так IPIP-NEO-300 (300 вопросов) и IPIP-NEO-120 (120 вопросов). Один из последних методов измерения большой пятерки является BFI-2, разработанная Оливером Джоном и Кристофером Сото в 2015 году \cite{big_five_inventory}.

\section{Типология Майерс-Бриггс}

\textbf{Myers--Briggs Type Indicator (MBTI)} --- это психологическая типология личности, разработанная Изабель Бриггс Майерс и ее матерью Кэтрин как дополнение типологии Карла Юнга.

MBTI широко используется в сфере личностного развития, управления персоналом, образования и командного взаимодействия.

Главная идея MBTI заключается в том, что любой вид человеческой личности можно описать используя один из 16 типов. В основе этого лежат 4 характеристики, каждая из которых имеет 2 противоположности \cite{Pittenger1993MeasuringMBTI}: 

\begin{enumerate}
	\item \textbf{Экстраверсия (E) --- Интроверсия (I)} --- показывает, откуда человек получает энергию. Экстраверты направлены на внешний мир --- им комфортно в общении, они заряжаются энергией от взаимодействия с другими. Интроверты же сосредоточены на внутреннем мире мыслей и ощущений, предпочитают одиночество и более глубокие размышления.
	
	\item \textbf{Сенсорика (S) --- Интуиция (N)} --- определяет, как человек воспринимает информацию. Сенсорики фокусируются на конкретных фактах, деталях и реальности настоящего момента. Интуиты обращают внимание на связи, возможности и смыслы, предпочитая абстрактное мышление и прогнозирование.
	
	\item \textbf{Мышление (T) --- Чувство (F)} --- описывает, каким образом человек принимает решения. Тип \textbf{T} руководствуется логикой, объективными критериями и рациональным анализом. Тип \textbf{F} ориентируется на ценности, эмоции и влияние решений на других людей.
	
	\item \textbf{Суждение (J) --- Восприятие (P)} --- характеризует отношение к структуре и планированию. Люди с предпочтением \textbf{J} стремятся к организованности, планам и завершённости. Люди с предпочтением \textbf{P} гибки, открыты к новым возможностям и чаще импровизируют.
\end{enumerate}

Комбинация четырёх харатеристик образует 16 типов личности, например: {INTJ, ENFP, ISFJ, ESTP} и т.д.  
Каждый тип представляет уникальный способ восприятия, мышления и взаимодействия с миром.

Например:

\begin{itemize}
	\item \textbf{INTJ} — стратег, склонный к анализу и планированию, предпочитающий независимость и долгосрочные цели.
	\item \textbf{ENFP} — энтузиаст, вдохновляющий других, гибкий и креативный в общении и идеях.
	\item \textbf{ISTJ} — надёжный и системный исполнитель, ценящий порядок и ответственность.
\end{itemize}

\begin{table}[!h]
	\begin{center}
		\small
		\begin{threeparttable}
			\caption{Сравнение подходов к описанию персональных черт}
			\label{tbl:exist_sol}
			\begin{tabular}{|p{5cm}|c|c|c|c|c|}
				\hline
				Критерий сравнения & Оллпорт & Айзенк & Кэттел & \makecell{Большая \\ пятёрка} & MBTI   \\
				\hline
				Применимость в моделировании поведения & - & - & - & + & - \\ 
				\hline
				Надёжность & Высокая & Высокая & Средняя & Высокая & Низкая \\ 
				\hline
				Вариативность моделируемых личностей & Высокая & Высокая & Высокая & Высокая & \makecell{Ограничена \\ 16 типами} \\ 
				\hline
				Наличие теста или опросника для определения типа личности & - & + & + & + & + \\
				\hline
			\end{tabular}
		\end{threeparttable}			
	\end{center}
\end{table}

Все подходы к описанию личностных черт, за исключением MBTI, позволяют описать произвольное число личностей (персонажей) на основе варьирования значений конкретных параметров. Так, конкретная черта (параметр модели) может быть задана как бинарная, тернарная (значения: низкое, среднее, высокое), на основе иной шкалы (например, с 7 делениями~\cite{}), на основе соотношения вклада черт в характер (пример: 10~\% той-то черты, ...соотношение, можно со ссылкой). %это отчет о какой-то строке таблицы


% обобщить обзор решений, применяющих конкретные подходы к описанию персональных черт (ПОПЧ) в виде таблице решения x подходы. Это ещё и классификация рассмотренных решений (программ, роботов, статей, описывающих исследования о моделировании поведения с учетом персональных черт) по ПОПЧ (например, в статье с Kismet-подобной головой явно применен подход на основе описанных И.Павловым свойств нервной системы: амплитуда реакций, сила реакций)

Если бы это было деревом, было бы так:
- На основе фундаментальных свойств нервной системы (на основе типологии Павлова)
- На основе сочетания черт, выделенных психологами
-- Большая пятёрка черт, Олпорт, Айзенк*
% БПЧ сама по себе (агрегированный, гибридный подход)
% Олпорт
% Айзенк
-- Кэттел
- (Иное) Недетерминированный (статистический?? но статистика может быт ьв разных подходах, вопрос лишь в наличии/отсутствии блока принятия решений непосредственно перед использованием статистических данных) подход на основе воспроизведения конкретных элементов поведения на основе статистических данных об их использовании конкретным прототипом персонажа: "персонажи" Алисы (голос + типовые фразы в некотором кол-ве --- они м.б. предварительно записаны диктором, а затем воспроизведены или сымитированы), дипфейки сюда же
%"недетерминированное": чаще всего нет закона в программе, который бы описывал целеориентированное и/или обусловленное влиянием персональных черт поведение (а не просто в виде правила if (вход = <шаблон в навыке навигатора Алисы>) then <вывести фразу спортивного комментатора>)
% Применимость есть для воспроизведения речевых паттернов, свойственных какому-то персонажу (присказки "-дон", "пу-пу-пу", мимика и жесты - специфичное подмигивание, междометия, подкручивание усов**), но это чисто внешняя сторона процесса порождения речи, а глубинные процессы, влияющие на принятие решения о выборе той или иной стратегии (или чуть ниже уровнем - сценария) поведения, нельзя смоделировать таким образом

% в презентации дерево могло бы быть хорошей визуализацией предложенной в данном НИРе классификации подходов к описанию персональных черт

* Айзенк и к биологии, и к психологии

** Те же элементы поведения можно выбирать на основе разных механизмов. И есть разница в том, выбран ли элемент поведения "подмигнуть" по логике "в корпусе записей реакций персонаж Х подмигиавает в среднем раз в две минуты или же по логике "будем идти по паттерну <<заговорщицкое, обозначить, что по шкале свой-чужой собеседник помещается на полюс "свой">>. Последний, детерминированный, принцип отбора стратегий/сценариев/элементов коммуникативного поведения куда с меньшей вероятностью выдаст неуместную реакцию (что так или иначе оценивается респондентами), чем "слепая" статистика в недетерминированном подходе
