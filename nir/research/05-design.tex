\chapter{Робот Ф-2}

\textbf{Робот Ф-2} -- исследовательский проект, в рамках которого изучаются способы как сделать роботов более привлекательными для человека. Робот способен воспроизводить эмоциональные и рациональные реакции, используя жесты, речь, мимику. 

Робот состоит из следующих основных компонент (в рамках данной работы не рассматриваются компоненты, связанные с компьютерным зрением и тактильным восприятием) \cite{arch_of_the_robot}:

\begin{itemize}
	\item лингвистический;
	\item компонент сценариев;
	\item компонент управления.
\end{itemize}


% картинка архитектуры


\section{Лингвистический компонент}

Для того, чтобы Ф-2 мог корректно среагировать на входящее сообщение на естественном языке (текстовое или аудио) необходимо провести соответствующую обработку сообщения. Это и есть задача лингвистического компонента (ЛК).

ЛК разделен на 3 модуля, которые работают последовательно \cite{arch_of_the_robot}. 

\begin{enumerate}
	\item \textbf{Морфологический} --- каждой словоформе в тексте приписываются морфологические и семантические признаки \cite{arch_of_the_robot_2}. В основе работы данного модуля лежит словарь из 100 тысяч лексем на основе проекта OpenCorpora \cite{arch_of_the_robot_3}.
	
	\item \textbf{Синтаксический} --- для каждого предложения строятся одно или несколько деревьев синтаксических зависимостей. Для этого используется словарь формализованных правил русского языка на языке syntXML.
	
	\item \textbf{Семантический} --- для каждого синтаксического дерева строится его семантическое представление: узлам дерева назначаются определенные валентности (роли слов в предложении) и семантические признаки. %, связанные с валентностью.
\end{enumerate}


\clearpage

\section{Компонент сценариев}

Поступающее на вход семантическое представление сравнивается с набором сценариев (д-сценарии и p-сценарии). Д-сценарии есть доминантные сценарии -- базовые аналоги эмоций, p-сценарии -- рациональные сценарии реагирования \cite{arch_of_the_robot_4}. 

Каждый сценарий включает аналогичные семантические структуры -- множества признаков, распределённых по валентностям. %это зеленое представление!
Следовательно, для каждой пары вида <семантическое представление, сценарий> вычисляется мера близости, она зависит от числа совпавших семантических признаков в тождественных валентностях. На основе меры близости и чувствительности сценариев вычисляется активизация каждого сценария \cite{arch_of_the_robot_5}. Сценарий включает внутреннее представление смысла входного события и коммуникативную цель (КЦ) ответа на него.

Сценарий, получивший при активации наибольший вес, выбирается для реализации, Для КЦ ответа из базы реакций выбирается произвольная реакция из числа сопоставленных данной КЦ. Мультимодальная реакция включает жесты, элементы мимики и текст, оформленные в формате BML (Behavior Markup Language). Кадр поведения  передается в компонент управления для исполнения на роботе.

\section{Компонент управления}

Компонент управления отвечает за выполнение поступивших BML-пакетов с помощью жестов, мимики, речи \cite{arch_of_the_robot}. Текстовый сегмент поведенческого кадра .....
Жестовая составляющая выполняется на основе перемещения сервомоторов робота по контрольным точкам, указанным в пакете команд, описывающих жест [Аринкин Н.А. ТЕМА. ВКР специалиста, специальность 09..... <<Программная инженерия>>. МГТУ им. Н. Э. Баумана, 2016. --- N c.68]