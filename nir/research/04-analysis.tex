\chapter{Аффективная робототехника}

Развитие роботизированных технологий влечёт за собой неизбежное возникновение вопросов, связанных с взаимодействием человека и робота, а также с восприятием робота как полноценного участника этого взаимодействия.

% что то мб про то, что люди склонны взаимодействовать с роботами как с людьми

\textbf{Аффективная робототехника} -- сфера деятельности, которая является пересечением между областями аффективных вычислений и человеко-машинной коммуникации. Она находит применение в различных областях, таких как: 

\begin{itemize}
	\item \textbf{медицина:} использование роботов для ухода за пожилыми людьми \cite{affect-rob-in-med, affect-rob-in-med-2};
	
	\item \textbf{образование:} роботы помогают в обучении детей, моделировании явлений, дистанционном участии в уроках и поддержке учителей, выполняя вспомогательные и интерактивные функции \cite{affect-rob-in-edu};
	
	\item \textbf{работа с клиентами:} роботы всё чаще применяются в сферах с активным человеческим взаимодействием — например, в аэропортах, музеях, отелях и ресторанах. \cite{affect-rob-in-serv};
	
	\item и др. \cite{affect-rob-in-other-fields}.
\end{itemize}


\textbf{Цель аффективной робототехники} -- приблизить машину к человеческому опыту и восприятию, позволяя роботам обмениваться эмоциональной информацией с людьми. Это включает в себя способность распознавать и интерпретировать человеческие эмоции, а также проявлять социальное и эмоциональное поведение \cite{affect-rob-goal}.


