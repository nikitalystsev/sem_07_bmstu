\chapter{Аффективная робототехника}

Развитие роботизированных технологий влечёт за собой неизбежное возникновение вопросов, связанных с взаимодействием человека и робота, а также с восприятием робота как полноценного участника этого взаимодействия.

% что то мб про то, что люди склонны взаимодействовать с роботами как с людьми

\textbf{Аффективная робототехника} --- сфера деятельности, которая является пересечением между областями аффективных вычислений и человеко-машинной коммуникации. Она находит применение в следующих областях. 

\begin{itemize}
	\item \textbf{Медицина:} использование роботов для ухода за пожилыми людьми \cite{affect-rob-in-med, affect-rob-in-med-2}.
	
	\item \textbf{Образование:} роботы помогают в обучении детей, моделировании явлений, дистанционном участии в уроках и поддержке учителей, выполняя вспомогательные и интерактивные функции \cite{affect-rob-in-edu}.
	
	\item \textbf{Работа с клиентами:} роботы всё чаще применяются в сферах с активным человеческим взаимодействием — например, в аэропортах, музеях, отелях и ресторанах \cite{affect-rob-in-serv}.
	
	\item и др. \textcolor{red}{?????} \cite{affect-rob-in-other-fields}.
\end{itemize}


\textbf{Цель аффективной робототехники} -- приблизить взаимодействие человека с машиной к человеческому опыту и восприятию, полученному в межличностном взаимодействии. В частности, средством достижения этой цели может быть введение такой функциональности роботов, которая позволит роботам обмениваться с людьми не только информацией, но и аффективной составляющей её восприятия --- аналогом эмоциональной окраски информации человеком или его отношения к ней. Для формирования такого человеко-машинного взаимодействия потребуется придать роботам способность распознавать и интерпретировать человеческие эмоции и/или проявлять социальное и эмоциональное поведение \cite{affect-rob-goal}.


