\chapter{Технологический раздел}

В данном разделе будут перечислены требования к программному обеспечению, средства реализации и листинги кода.

\section{Требования к программному обеспечению}

К программе предъявляется ряд требований:

\begin{itemize}
	\item программа должна предоставлять возможность сжатия и разжатия произвольного файла;
	\item программа должна рассчитывать коэффициент сжатия;
	\item программа должна корректно работать с пустым однобайтовым файлом;
\end{itemize}

\section{Средства реализации}

В качестве языка программирования для этой лабораторной работы был выбран $C++$ \cite{pl}.

\section{Сведения о модулях программы}

Программа состоит из трех модулей: 

\begin{itemize}
	\item \texttt{main.cpp} --- файл, содержащий точку входа в программу;
	\item \texttt{LZW.cpp} --- модуль, реализующий алгоритм LZW;
	\item \texttt{TrieTree.cpp} --- модуль, реализующий префиксное дерево.
\end{itemize}

\section{Тестирование}

%\begin{table}[H]
%	\begin{center}
%		\begin{threeparttable}
%			\caption{Функциональные тесты}
%			\label{tbl:cmpResponseTimeByRequests_s_withoutCache}
%			\begin{tabular}{|r|r|}
%				\hline
%				\bfseries \makecell{Входная строка} & \bfseries \makecell{Шифрованная строка}\\
%				\hline
%				Hello & g▓ùδÄ☺v┼  \\ 
%				\hline
%				g▓ùδÄ☺v┼  & Hello  \\ 
%				\hline
%				B & h∩▐í'(Ñ!  \\ 
%				\hline
%				h∩▐í'(Ñ! & B  \\ 
%				\hline
%			\end{tabular}
%		\end{threeparttable}
%	\end{center}
%\end{table}

Все тесты с файлами были успешно пройдены.

\section{Реализации алгоритмов}

На листинге \ref{lst:lzw-compress.txt} представлена функция, реализующая сжатие по алгоритму LZW.


\includelisting
{lzw-compress.txt} % Имя файла с расширением (файл должен быть расположен в директории inc/lst/)
{Функция, реализующая сжатие по алгоритму LZW} % Подпись листинга

\clearpage

\includelisting
{lzw-decompress-part1.txt} % Имя файла с расширением (файл должен быть расположен в директории inc/lst/)
{Функция, реализующая разжатие по алгоритму LZW (начало)} % Подпись листинга

\clearpage 

\includelisting
{lzw-decompress-part2.txt} % Имя файла с расширением (файл должен быть расположен в директории inc/lst/)
{Функция, реализующая разжатие по алгоритму LZW (конец)} % Подпись листинга

\section*{Вывод}

В данном разделе были перечислены требования к программному обеспечению, средства реализации и листинги кода.

    