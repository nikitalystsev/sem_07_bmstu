\chapter{Аналитическая часть}


\textbf{LZW} -- алгоритм сжатия, основывающийся на поиске схожих символов в файле.

Алгоритм состоит из следующих шагов:

\textbf{Кодирование}

\begin{enumerate}
	\item  Все возможные символы заносятся в словарь. Во входную фразу X
	заносится первый символ сообщения.
	\item Считать очередной символ Y
	из сообщения.
	\item Если Y --- это символ конца сообщения, то выдать код для X, иначе:
	\item Если фраза XY уже имеется в словаре, то присвоить входной фразе значение XY и перейти к Шагу 2,
	\item Иначе выдать код для входной фразы X, добавить XY в словарь и присвоить входной фразе значение Y. Перейти к Шагу 2.
\end{enumerate}

\textbf{Декодирование}

\begin{enumerate}
	\item Все возможные символы заносятся в словарь. Во входную фразу X
	заносится первый код декодируемого сообщения.
	\item Считать очередной код Y из сообщения.
	\item Если Y --- это конец сообщения, то выдать символ, соответствующий коду X, иначе:
	\item Если фразы под кодом XY нет в словаре, вывести фразу, соответствующую коду X, а фразу с кодом XY занести в словарь.
	\item Иначе присвоить входной фразе код XY и перейти к Шагу 2.
\end{enumerate}


\section*{Вывод} 

В этом разделе был рассмотрен алгоритм сжатия LZW.