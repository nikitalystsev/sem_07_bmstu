\chapter{Аналитическая часть}

В этом разделе будут рассмотрен криптографический алгоритм RSA, алгоритм хеширования SHA1 понятие электронной подписи и принципы её получения и проверки с использованием алгоритмов RSA, SHA1.

\section{Электронная цифровая подпись}

Электронная цифровая подпись (ЭЦП) позволяет подтвердить авторство электронного документа.

Создание ЭЦП с использованием криптографического алгоритма RSA и алгоритма хеширования SHA1 происходит следующим образом:
\begin{enumerate}[label=\arabic*)]
	\item файл, который необходимо подписать, хешируется  при помощи SHA1;
	\item полученный на предыдущем этапе хеш шифруется с использованием закрытого ключа RSA;
	\item значение подписи --- результат шифрования.
\end{enumerate}

Проверка подлинности ЭЦП с использованием криптографического алгоритма RSA и алгоритма хеширования SHA1 происходит следующим образом:

\begin{enumerate}[label=\arabic*)]
	\item файл, который необходимо подписать, хешируется  при помощи SHA1;
	\item переданная подпись расшифровывается с использованием открытого ключа RSA;
	\item происходит побитовая сверка значений, полученных на предыдущих этапах, если они одинаковы, подпись считается подлинной.
\end{enumerate}

\section{Алгоритм RSA}

RSA (аббревиатура от фамилий \textit{Rivest}, \textit{Shamir} и \textit{Adleman}) --- ассиметричный алгоритм с открытым ключом, основывающийся на вычислительной сложности задачи факторизации больших полупростых чисел. В алгоритме RSA используется 2 ключа --- открытый (публичный) и закрытый (приватный).

В ассиметричной криптографии и алгоритме RSA, в частности, открытый и закрытый ключи являются двумя частями одного целого и неразрывны друг с другом. Для шифрования информации используется открытый ключ, а для её расшифровки закрытый.

Криптосистема RSA стала первой системой, пригодной и для шифрования, и для цифровой подписи.

Открытый и закрытый ключи RSA генерируются следующим образом:

\begin{enumerate}[label=\arabic*)]
	\item выбираются два разных случайных простых числа $p$ и $q$ заданного размера;
	\item вычисляется их произведение $n = p \cdot q$, называемое модулем;
	\item вычисляется значение функции Эйлера от числа $n$ по следующей формуле:
	
	$$\phi(n) = (p - 1)\cdot (q - 1)$$
	
	\item выбирается целое число $e$ ($1 < e < \phi(n)$), взаимно простое со значением $\phi(n)$, называемое \textit{открытой экспонентой};
	
	\item вычисляется число $d$, называемое \textit{закрытой экспонентой} по следующей формуле:
	
	$$d  = e^{-1} mod (\phi(n))$$
	
	\item пара $(e, n)$ публикуются в качестве открытого ключа RSA;
	\item пара $(d, n)$  представляет собой закрытый ключ RSA.
\end{enumerate}

Перевод исходного сообщения $m$ в зашифрованное сообщение $c$ производится по формуле

 $$c = E(m, k_1) = E(m, n, e) = m^{e} mod (n)$$

Расшифровка выполняется по формуле

$$m = D(c, k_2) = D(c, n, d) = c^{d} mod (n)$$

\includeimage
{rsa}
{f}
{ht!} 
{0.8\textwidth}
{RSA}

\section{Алгоритм SHA1}

SHA1 (англ. \textit{Secure Hash Algorithm 1}) --- алгоритм криптографического хеширования. 
Для входного сообщения произвольной длины алгоритм генерирует 160-битное (20 байт) хеш-значение, называемое также дайджестом сообщения, которое обычно отображается как шестнадцатеричное число длиной в 40 цифр.

Алгоритм состоит из следующих шагов:

\begin{enumerate}
	\item \textbf{добавление недостающих битов}. Сообщение расширяется таким образом, чтобы его длина была кратна 448 по модулю 512 (длина $\equiv  448 \mod 512$). Добавление осуществляется всегда, даже если сообщение уже имеет нужную длину. Таким образом, число добавляемых битов находится в диапазоне от 1 до 512;
	\item \textbf{расширение.} Результатом первых двух шагов является сообщение, длина которого кратна 512 битам. Расширенное сообщение может быть представлено как последовательность 512-битных блоков $Y_0, Y_1, . . . , Y_{L-1}$, так что общая длина расширенного сообщения есть $L * 512$ бит. Таким образом, результат кратен шестнадцати 32-битным словам.
	\item \textbf{инициализация SHA-1 буфера.} Используется 160-битный буфер для хранения промежуточных и окончательных результатов хэш-функции. Буфер может быть представлен как пять 32-битных регистров A, B, C, D и E. Эти регистры инициализируются следующими шестнадцатеричными числами: A = 67452301, B = EFCDAB89, C = 98BADCFE, D = 10325476, E = C3D2E1F0.
	\item \textbf{обработка сообщения в 512-битных (16-словных) блоках.} Основой алгоритма является модуль, состоящий из 80 циклических обработок, обозначенный как $H_{SHA}$. Все 80 циклических обработок имеют одинаковую структуру.
	\item \textbf{выход.} После обработки всех 512-битных блоков выходом L-ой стадии является 160-битный дайджест сообщения.
\end{enumerate}

\includesvgimage
{SHA-1}
{f}
{ht!} 
{0.6\textwidth}
{Одна итерация алгоритма SHA-1}

\section*{Вывод} 

В этом разделе был рассмотрен криптографический алгоритм RSA, алгоритм хеширования SHA1 понятие электронной подписи и принципы её получения и проверки с использованием алгоритмов RSA, SHA1.