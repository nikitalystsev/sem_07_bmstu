\chapter{Конструкторский раздел}

На рисунках \ref{img:rsa-}-\ref{img:sha1-iter} представлены схемы алгоритма RSA и SHA1.

\includeimage
{rsa-} % Имя файла без расширения (файл должен быть расположен в директории inc/img/)
{f} % Обтекание (без обтекания)
{ht!} % Положение рисунка (см. figure из пакета float)
{0.8\textwidth} % Ширина рисунка
{Схема шифровального алгоритма RSA} % Подпись рисунка

\includeimage
{rsa-work.png} % Имя файла без расширения (файл должен быть расположен в директории inc/img/)
{f} % Обтекание (без обтекания)
{ht!} % Положение рисунка (см. figure из пакета float)
{0.7\textwidth} % Ширина рисунка
{Схема шифровального алгоритма работы RSA} % Подпись рисунка

\includeimage
{sha-1.png} % Имя файла без расширения (файл должен быть расположен в директории inc/img/)
{f} % Обтекание (без обтекания)
{ht!} % Положение рисунка (см. figure из пакета float)
{0.8\textwidth} % Ширина рисунка
{Схема алгоритма SHA1} % Подпись рисунка

\includeimage
{sha1-iter.png} % Имя файла без расширения (файл должен быть расположен в директории inc/img/)
{f} % Обтекание (без обтекания)
{ht!} % Положение рисунка (см. figure из пакета float)
{0.8\textwidth} % Ширина рисунка
{Схема раунда алгоритма SHA1} % Подпись рисунка

\section*{Вывод}

В данном разделе были представлены схемы алгоритма RSA и SHA1.





