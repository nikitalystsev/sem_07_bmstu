\chapter{Технологический раздел}

В данном разделе будут перечислены требования к программному обеспечению, средства реализации и листинги кода.

\section{Требования к программному обеспечению}

К программе предъявляется ряд требований:

\begin{itemize} [label=--]
	\item программа должна предоставлять возможность шифрования и расшифровки произвольного файла, а также текстового сообщения;
	\item программа должна корректно работать с пустым однобайтовым файлом;
	\item программа должна уметь обрабатывать файл архива.
\end{itemize}

\section{Средства реализации}

В качестве языка программирования для этой лабораторной работы был выбран $C++$ \cite{pl}.

\section{Сведения о модулях программы}

Программа состоит из трех модулей: 

\begin{itemize}
	\item \texttt{main.cpp} --- файл, содержащий точку входа в программу;
	\item \texttt{AES.cpp} --- модуль, реализующий алгоритм AES;
	\item \texttt{CFB.cpp} --- модуль, реализующий шифрование в режиме CFB.
\end{itemize}

\section{Тестирование}

\begin{table}[H]
	\begin{center}
		\begin{threeparttable}
			\caption{Функциональные тесты}
			\label{tbl:cmpResponseTimeByRequests_s_withoutCache}
			\begin{tabular}{|r|r|}
				\hline
				\bfseries \makecell{Входная строка} & \bfseries \makecell{Шифрованная строка}\\
				\hline
				Hello & g▓ùδÄ☺v┼  \\ 
				\hline
				g▓ùδÄ☺v┼  & Hello  \\ 
				\hline
				B & h∩▐í'(Ñ!  \\ 
				\hline
				h∩▐í'(Ñ! & B  \\ 
				\hline
			\end{tabular}
		\end{threeparttable}
	\end{center}
\end{table}

Все тесты были успешно пройдены. Тесты с файлами были также успешно пройдены.

\section{Реализации алгоритмов}

На листинге \ref{lst:aes.txt} представлена функция, реализующая алгоритм AES.

На листинге \ref{lst:key-expansion.txt} представлена функция, реализующая генерацию ключей для каждого раунда шифрования.

На листинге \ref{lst:add-round-key.txt}  представлена функция, реализующая добавление ключа раунда к state.

На листинге \ref{lst:sub-bytes.txt}  представлена функция, реализующая нелинейную замену байт в state с помощью sbox.

На листинге \ref{lst:shift-rows.txt}  представлена функция, реализующая циклический сдвиг строк state влево в зависимости от номера строк.

На листинге \ref{lst:mix-columns.txt}  представлена функция, реализующая процедуру MixColumns.

%На листингах \ref{lst:f_part1.txt} и \ref{lst:f_part2.txt} представлена реализация функции Фейстеля.
%
%На листинге \ref{lst:pcbc.txt}  представлена реализация шифрования в режиме PCBC.
%
\includelisting
{aes.txt} % Имя файла с расширением (файл должен быть расположен в директории inc/lst/)
{Функция, реализующая алгоритм AES} % Подпись листинга

\clearpage

\includelisting
{key-expansion.txt} % Имя файла с расширением (файл должен быть расположен в директории inc/lst/)
{Функция, реализующая генерацию ключей раундов} % Подпись листинга

\includelisting
{add-round-key.txt} % Имя файла с расширением (файл должен быть расположен в директории inc/lst/)
{Функция, реализующая добавление ключа раунда к state} % Подпись листинга

\clearpage

\includelisting
{sub-bytes.txt} % Имя файла с расширением (файл должен быть расположен в директории inc/lst/)
{Функция, реализующая нелинейную замену байт в state с помощью sbox} % Подпись листинга

\includelisting
{shift-rows.txt} % Имя файла с расширением (файл должен быть расположен в директории inc/lst/)
{Функция, реализующая циклический сдвиг строк state влево в зависимости от номера строки} % Подпись листинга

\includelisting
{mix-columns.txt} % Имя файла с расширением (файл должен быть расположен в директории inc/lst/)
{Функция, реализующая процедуру MixColumns} % Подпись листинга

%\includelisting
%{f_part1.txt} % Имя файла с расширением (файл должен быть расположен в директории inc/lst/)
%{Реализация функции Фейстеля (начало)} % Подпись листинга
%
%\includelisting
%{f_part2.txt} % Имя файла с расширением (файл должен быть расположен в директории inc/lst/)
%{Реализация функции Фейстеля (конец)} % Подпись листинга
%
%\includelisting
%{pcbc.txt} % Имя файла с расширением (файл должен быть расположен в директории inc/lst/)
%{Реализация шифрования в режиме PCBC} % Подпись листинга
%\clearpage
%
%В листинге \ref{lst:encrypt-char-by-code-impl.txt} представлена реализация алгоритма шифрования символа.
%
%\includelisting
%{encrypt-char-by-code-impl.txt} % Имя файла с расширением (файл должен быть расположен в директории inc/lst/)
%{Реализация шифрования символа} % Подпись листинга
%
%В листинге \ref{lst:encrypt-impl.txt} представлена реализация алгоритма шифрования символа c использованием кодировщика.
%
%\includelisting
%{encrypt-impl.txt} % Имя файла с расширением (файл должен быть расположен в директории inc/lst/)
%{Реализация шифрования символа с использованием кодировщика} % Подпись листинга
%
%В листингах \ref{lst:encrypt-string-impl-part1.txt}  и \ref{lst:encrypt-string-impl-part2.txt} представлена реализация шифрования строки.
%
%\includelisting
%{encrypt-string-impl-part1.txt} % Имя файла с расширением (файл должен быть расположен в директории inc/lst/)
%{Реализация шифрования строки (начало)} % Подпись листинга
%
%\includelisting
%{encrypt-string-impl-part2.txt} % Имя файла с расширением (файл должен быть расположен в директории inc/lst/)
%{Реализация шифрования строки (конец)} % Подпись листинга
%
%В листингах \ref{lst:encrypt-file-impl-part1.txt}  и \ref{lst:encrypt-file-impl-part2.txt} представлена реализация шифрования произвольного файла.
%
%\includelisting
%{encrypt-file-impl-part1.txt} % Имя файла с расширением (файл должен быть расположен в директории inc/lst/)
%{Реализация шифрования произовльного файла (начало)} % Подпись листинга
%
%\includelisting
%{encrypt-file-impl-part2.txt} % Имя файла с расширением (файл должен быть расположен в директории inc/lst/)
%{Реализация шифрования произовльного файла (конец)} % Подпись листинга
%
%%\includelisting
%%{encrypt-string-impl-part2.txt} % Имя файла с расширением (файл должен быть расположен в директории inc/lst/)
%%{Реализация шифрования строки (конец)} % Подпись листинга
%
%%В листингах \ref{lst:search-tree-part1.txt} и \ref{lst:search-tree-part2.txt} представлена реализация функции поиска в ДДП и АВЛ-дереве.
%%
%%\includelisting
%%{search-tree-part1.txt} % Имя файла с расширением (файл должен быть расположен в директории inc/lst/)
%%{Реализация функции поиска в ДДП и АВЛ-дереве (начало)} % Подпись листинга
%%
%%\includelisting
%%{search-tree-part2.txt} % Имя файла с расширением (файл должен быть расположен в директории inc/lst/)
%%{Реализация функции поиска в ДДП и АВЛ-дереве (конец)} % Подпись листинга

\section*{Вывод}

В данном разделе были перечислены требования к программному обеспечению, средства реализации и листинги кода.

    