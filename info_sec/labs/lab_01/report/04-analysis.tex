\chapter{Аналитический раздел}

В данном разделе  будет формально описано устройство шифровальной машины <<Энигма>>, а также кратко будет описан алгоритм ее работы.

\section{Шифровальная машина <<Энигма>>}

\textbf{Энигма} -- криптографическая машина, созданная немецкими военными для обеспечения своих коммуникаций \cite{info_Enigma1}. В ее основе лежали 3 основных механизма:

\begin{itemize}
	\item \textit{роторы} -- наборы вращающихся дисков. На каждом диске было 26 граней, на каждой из которых была нанесена буква английского алфавита. Число роторов варьируется от трех до восьми \cite{info_Enigma1, info_Enigma2}. Они используются для преобразования одной буквы в другую; 
	\item \textit{рефлектор} -- статичный механизм, позволяющий не вводить дополнительную операцию расщифрования; 
	\item \textit{коммутатор} -- механизм, позволяющий оператору шифровальной машины соединять попарно одни буквы с другими.
\end{itemize}

\section{Алгоритм работы шифровальной машины <<Энигма>>}

Пусть Энигма состоит из трех роторов и  одного рефлектора, а также 26-ти соединительных проводов для коммутационной панели. Алгоритм работы Энигмы состоит из следующих шагов:

\begin{enumerate}
	\item на вход поступает текстовое сообщение;
	\item каждый символ поступает в коммутационную панель;
	\item определяется символ, парный данному;
	\item символ проходит через каждый ротор, где осуществляется преобразование в новый символ;
	\item после 3 роторов символ поступает в рефлектор и определяется символ, парный данному;
	\item полученный на предыдущем шаге символ в обратном направлении проходит через все роторы;
	\item новый символ поступает в коммутатор и определятся символ, парный данному;
	\item определенный на предыдущем шаге символ есть шифр исходного символа;
	\item первый ротор поворачивается на одну позицию. Если первый ротор совершил полный оборот, то на одну позицию поворачивается соседний ротор. Если второй ротор совершил полный оборот, то третий поворачивается на одну позицию. 
\end{enumerate}

\section*{Вывод}

В данном разделе было формально описано устройство шифровальной машины <<Энигма>>, а также был кратко описан алгоритм ее работы.