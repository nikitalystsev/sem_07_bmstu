\chapter{Графовые модели алгоритмов}

Программа представлена в виде графа --- набора вершин и множества соединяющих их дуг.
Вершины --- операторы, срабатывания операторов.
Дуги --- отношения.

Существует два типа отношений:

\begin{itemize}[label*=--]
	\item информационное отношение --- отношение по передаче данных;
	\item операционное отношение --- отношение по передаче управления.
\end{itemize}

Выделяют четыре  графовых модели:

\begin{itemize}[label*=--]
	\item \texttt{граф управления} --- модель, в которой вершинами являются операторы, а дугами --- операционные отношения;
	\item \texttt{информационный граф} --- модель, в которой вершинами являются операторы, а дугами --- информационные отношения;
	\item \texttt{операционная история} --- модель, в которой вершинами являются срабатывания операторов, а дугами --- операционные отношения;
	\item \texttt{информационная история} --- модель, в которой вершинами являются срабатывания операторов, а дугами --- информационные отношения.
\end{itemize}
